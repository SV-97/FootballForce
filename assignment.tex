%%% Template originaly created by Karol Kozioł (mail@karol-koziol.net) and modified for ShareLaTeX use

\documentclass[a4paper,11pt]{article}

\usepackage[german]{babel}
\usepackage[T1]{fontenc}
\usepackage[utf8]{inputenc}
\usepackage{graphicx}
\usepackage{xcolor}

\renewcommand\familydefault{\sfdefault}
%\usepackage{tgheros}
%\usepackage[defaultmono]{droidmono}

\usepackage{amsmath,amssymb,amsthm,textcomp}
\usepackage{enumerate}
\usepackage{multicol}
\usepackage{tikz}
\usepackage{pdfpages}
\usepackage{graphics}

\usepackage{geometry}
\geometry{left=15mm,right=15mm,%
bindingoffset=0mm, top=10mm,bottom=10mm}


\linespread{1.3}

\newcommand{\linia}{\rule{\linewidth}{0.5pt}}

% custom theorems if needed
\newtheoremstyle{mytheor}
    {1ex}{1ex}{\normalfont}{0pt}{\scshape}{.}{1ex}
    {{\thmname{#1 }}{\thmnumber{#2}}{\thmnote{ (#3)}}}

\theoremstyle{mytheor}
\newtheorem{defi}{Definition}
\newtheorem{problem}{Problem}

% my own titles
\makeatletter
\renewcommand{\maketitle}{
\begin{center}
\vspace{2ex}
{\huge \textsc{\@title}}
\vspace{1ex}
\\
\linia\\
\@author \hfill \@date
\vspace{4ex}
\end{center}
}
\makeatother
%%%

% custom footers and headers
\usepackage{fancyhdr}
\pagestyle{fancy}
\lhead{}
\chead{}
\rhead{}
%\lfoot{Assignment \textnumero{} 2}
\cfoot{}
% \rfoot{Page \thepage}
\renewcommand{\headrulewidth}{0pt}
\renewcommand{\footrulewidth}{0pt}
\newcommand{\impliesBecause}[1]{\underset{#1}
{\implies}}
\renewcommand{\implies}{\Rightarrow}
\renewcommand{\Re}[0]{\text{Re}}
\renewcommand{\Im}[0]{\text{Im}}
\newcommand{\norm}[1]{\|{#1}\|_2}
\newcommand{\spann}[1]{\text{span}(#1)}
\newcommand{\rang}[0]{\text{Rang}}
\newcommand{\comp}[0]{\text{comp}}
\newcommand{\atom}[0]{\text{atom}}
\newcommand{\diag}[0]{\text{diag}}
\newcommand{\diagm}[0]{\text{diagmat}}

\usepackage{mathtools}
\DeclarePairedDelimiter\ceil{\lceil}{\rceil}
\DeclarePairedDelimiter\floor{\lfloor}{\rfloor}
%

% code listing settings
\usepackage{listings}
\lstset{
    language=Python,
    basicstyle=\ttfamily\small,
    aboveskip={1.0\baselineskip},
    belowskip={1.0\baselineskip},
    columns=fixed,
    extendedchars=true,
    breaklines=true,
    tabsize=4,
    prebreak=\raisebox{0ex}[0ex][0ex]{\ensuremath{\hookleftarrow}},
    frame=lines,
    showtabs=false,
    showspaces=false,
    showstringspaces=false,
    keywordstyle=\color[rgb]{0.627,0.126,0.941},
    commentstyle=\color[rgb]{0.133,0.545,0.133},
    stringstyle=\color[rgb]{01,0,0},
    numbers=left,
    numberstyle=\small,
    stepnumber=1,
    numbersep=10pt,
    captionpos=t,
    escapeinside={\%*}{*)}
}

%%%----------%%%----------%%%----------%%%----------%%%

\begin{document}

\title{Fußball Reibung}

\author{Simon, Volz}

\date{\today}

\maketitle

\begin{problem}
Ein Fußballspieler schießt einen Fußball vom Sportplatz aus ab und dieser trifft das Dach eines benachbarten Hauses. Mit welcher Kraft trifft der Ball das Dach?
\end{problem}

Um die Kraft (oder evt. geeigneter Impuls, Energie o.ä.) zu bestimmen, mit welcher der Ball das Dach trifft wollen wir zunächst bestimmen an welchem Punkt und mit welcher Geschwindigkeit der Ball das Dach trifft. Hierzu vernachlässigen wir zunächst eventuelle Einflüsse durch Wind, berücksichtigen jedoch die Luftreibung des Balls sowie die Wirkung der Schwerkraft. Da $s = \int v(t) dt$ ist die Bestimmung (ggf. durch numerische Integration) von $s$ einfach sobald wir $v$ kennen.
Sei $v(t) =(v_x(t), v_y(t))^T$ die Geschwindigkeit des Balls zur Zeit t und $\tan \alpha = \frac{v_y}{v_x}$. Die Luftreibungskraft $F_R$ wirkt stets engegen der Bewegungsrichtung des Balls. Wir teilen diese Reibungskraft auf die $x,y$ Komponenten auf und es gilt für die Komponenten $F_x, F_y$ der auf den Ball wirkenden Kraft: $F_x = -F_R \cos \alpha, F_y = -(F_R \sin \alpha + F_G)$. Nach $F=ma$ folgt
für die $x$-Komponente $a_x$ der Beschleunigung $a$ des Balls (mit $k=\frac{1}{2}c_w \rho A, \kappa=\frac{k}{m}$)
\begin{align*}
    a_x        & = -\frac{F_R \cos \alpha}{m}                                      \\
    \iff -ma_x & = k \norm{v}^2 \cos \alpha                                        \\
    \iff 0     & = \kappa (v_x^2 + v_y^2) \cos \arctan \frac{v_y}{v_x} + \dot{v}_x
\end{align*}
und für die $y$-Komponente $a_y$
\begin{align*}
    a_y        & = -\frac{F_R \sin \alpha + F_G}{m}                                     \\
    \iff -ma_y & = k \norm{v}^2 \sin \alpha + m g                                       \\
    \iff 0     & = \kappa (v_x^2 + v_y^2) \sin \arctan \frac{v_y}{v_x} + g + \dot{v}_y.
\end{align*}
Es ist also das Differenzialgleichungssystem
\begin{align*}
    0 & = \kappa (v_x^2 + v_y^2) \cos \arctan \frac{v_y}{v_x} + \dot{v}_x     \\
    0 & = \kappa (v_x^2 + v_y^2) \sin \arctan \frac{v_y}{v_x} + \dot{v}_y + g
\end{align*}
mit der Randbedingung $v_x(0), v_y(0) \in \mathbb{R}$ zu lösen.
Es sei $$h := \left(
    \begin{array}{c}
            x       \\
            y       \\
            \dot{x} \\
            \dot{y} \\
        \end{array}
    \right) \implies \dot{h} = \left(
    \begin{array}{c}
            \dot{x}  \\
            \dot{y}  \\
            \ddot{x} \\
            \ddot{y} \\
        \end{array}
    \right)$$
mit $v_x = \dot{x}, v_y = \dot{y}$. Dann lässt sich das oben genannte AWP wie folgt umschreiben
$$
    \dot{h} = f(t,h) = f(t, (x, y, \dot{x}, \dot{y})^T) = \left( \begin{array}{c}
            \dot{x}                                                                 \\
            \dot{y}                                                                 \\
            -\kappa(\dot{x}^2 + \dot{y}^2) \cos \arctan \frac{\dot{y}}{\dot{x}}     \\
            -\kappa(\dot{x}^2 + \dot{y}^2) \sin \arctan \frac{\dot{y}}{\dot{x}} - g \\
        \end{array} \right) = \left( \begin{array}{c}
            \dot{x}                                                                                            \\
            \dot{y}                                                                                            \\
            -\frac{\kappa(\dot{x}^2 + \dot{y}^2)}{\sqrt{(\frac{\dot{y}}{\dot{x}})^2+1}}                        \\
            -(\frac{\kappa \dot{y} (\dot{x}^2 + \dot{y}^2)}{\dot{x} \sqrt{(\frac{\dot{y}}{\dot{x}})^2+1}} + g) \\
        \end{array} \right)
$$

\end{document}
